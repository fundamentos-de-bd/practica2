\documentclass[10pt]{article}
\usepackage[utf8]{inputenc}
\usepackage[spanish]{babel}
\usepackage[usenames,dvipsnames,svgnames,table]{xcolor}
\usepackage{multirow}
\usepackage{diagbox}
\usepackage{booktabs}
\usepackage{anysize} 
\usepackage{hyperref}
\usepackage{helvet}
\renewcommand\refname{Referencias}
\marginsize{2cm}{2cm}{2.0cm}{2cm}
\usepackage{enumitem}
\usepackage{setspace}


\hypersetup{
    colorlinks=true,
    linkcolor=blue,
    filecolor=magenta,
    urlcolor=cyan,
    citecolor=blue
}


\usepackage{chronosys}



\begin{document}
    \title{Fundamentos de Bases de Datos \\
        Practica 2\\ Manipulación de archivos
        } 
    \author{}
    \date{01 de Marzo del 2019}
    \maketitle
    
    \section{Implementación del programa}\vspace{0.5cm}
    En esta practica se hace un avance del caso de uso enfoncadonos unicamente en la información de las sucursales y empleados, así que se implementa  un programa que solicita la información y lo guarda en un archivo CSV.
    
    De acuerdo a los requerimientos solicitados en el caso de uso, para los empleados se almacena su información básica: nombre, apellido paterno, apellido materno, puesto salario y sucursal a la que pertenece, se pueden guardar empleados siempre y cuando esten asociados a una sucursal existente, además al registrar un empleado se le genera  un identificador.
    
    El programa permite agregar, eliminar y modificar sucursales o empleados, además tenemos
    la opción de buscar una sucursal mediante el identificador de un empleado.\\
    
    
    Para lograr esto se dicidio modelar el programa con las siguientes clases:
    
    \begin{enumerate}
    	\item {\bf{Sucursal}} \\
    	
    	Esta clase tiene el metodo toCSV() que permite que los archivos puedan ser leidos y escritos.
    	
    	\item {\bf{Empleado}}\\
    	
    	
    	\item {\bf{Manejador}}\\
    	
    	\item {\bf{Menu}}\\
    	
    	\item {\bf{IO}}\\
    	
    \end{enumerate} 

    \subsection{Diferencias entre almacenar la información utilizando un sistema de archivos  a almacenarla en una base de datos. }
    
    Se menciona 5 diferencias entre almacenar la información mediante un sitema de archivos a almacenarla en una base de datos:
    
    \begin{itemize}
    	\item Mientras mas entidades se tengan mas complicado se vuelve el manejo de las referencias.
    	\item
    \end{itemize}
    
\end{document}