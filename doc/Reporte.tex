\documentclass[10pt]{article}
\usepackage[utf8]{inputenc}
\usepackage[spanish]{babel}
\usepackage[usenames,dvipsnames,svgnames,table]{xcolor}
\usepackage{multirow}
\usepackage{diagbox}
\usepackage{booktabs}
\usepackage{anysize} 
\usepackage{hyperref}
\usepackage{helvet}
\renewcommand\refname{Referencias}
\marginsize{2cm}{2cm}{2.0cm}{2cm}
\usepackage{enumitem}
\usepackage{setspace}


\hypersetup{
    colorlinks=true,
    linkcolor=blue,
    filecolor=magenta,
    urlcolor=cyan,
    citecolor=blue
}





\begin{document}
    \title{Fundamentos de Bases de Datos \\
        Practica 2\\ Manipulación de archivos
        } 
    \author{}
    \date{01 de Marzo del 2019}
    \maketitle
    
    \section{Implementación del programa en Java}\vspace{0.5cm}
    En esta practica se hace un avance del caso de uso enfoncadonos unicamente en la información de las sucursales y empleados, así que se implementa  un programa que solicita la información y lo guarda en un archivo CSV.
    
    De acuerdo a los requerimientos solicitados en el caso de uso, para los empleados se almacena su información básica: nombre, apellido paterno, apellido materno, puesto salario y sucursal a la que pertenece, se pueden guardar empleados siempre y cuando esten asociados a una sucursal existente, además al registrar un empleado se le genera  un identificador.
    
    El programa permite agregar, eliminar y modificar sucursales o empleados, además tenemos
    la opción de buscar una sucursal mediante el identificador de un empleado.\\
    
    
    Para lograr esto se dicidio modelar el programa con las siguientes clases:
    
    \begin{enumerate}
    	\item {\bf{Sucursal}} \\
    	
    	La clase Sucursal simula la entidad sucursal, esta clase implementa la interfaz CSV. El constructor correspondiente a esta clase es  {\texttt{Sucursal}} que recibe un String que representa el nombre del objeto Sucursal y en dicho constructor inicializa un identificador cuyo valor depende de la variable estática proxNumSucursal que se incrementa auntomaticamente en uno al crear una nueva instancia. En seguida se mencionan los principales metodos de esta clase:
    	
    	\begin{itemize}
    		
    		\item {\texttt{toCSV()}}. Este método permite regresar una representación de la instancia Sucursal en formato CSV.
    	\end{itemize} 
    	
    	
    	
    	\item {\bf{Empleado}}\\
    	
    	Al igual que la clase Sucursal, la clase Empleado modela la entidad empleado, también implementa la interfaz CSV. El constructor para esta clase recibe varios parametros relacionados con la información básica de un empleado, de igual forma se inicializa un identificador unico a partir de la variable estatica proxNumEmpleado que se va incrementando en uno de forma automatica al crear una nueva instancia.
    	Los métodos principales para esta clase son:\\
    	
    	\begin{itemize}
    		\item {\texttt{toCSV()}}. Método que regresa una representación en formato CSV de la instancia Empleado.
    	\end{itemize}
    	
    	\item {\bf{Manejador}}\\
    	
    	La clase Manejador, se diseño con el fin tener una clase que se encargue de administrar los datos del empleado y la sucursal por medio de  tablas Hash que son una estructura de datos implementada en Java por la clase HashMap, por lo tanto se tendrá una tabla hash para empleados y otra para sucursales.    Esta clase contiene los métodos principales de nuestro programa, que es el de agregar, eliminar, modificar y buscar por número de empleado.
    	\begin{itemize}
    		\item {\texttt{Manejador()}}.Corresponde al constructor de la clase que inicializa las tablas hash.
    		\item {\texttt{agregaSucursal()}}. Agrega una instancia de Sucursal a la tabla hash sucursales.
    		\item {\texttt{agregaEmpleado()}}. Agrega una instancia de Empleado a la tabla hash empleados.
    		\item {\texttt{eliminaSucursal()}}. Elimina una sucursal usando el numero de identificador de la sucursal.
    		\item {\texttt{eliminaEmpleado()}}. Elimia un empleado usando el numero de identificador del empleado.
    		\item {\texttt{modificaSucursal()}}.Este método modifica el nombre de la sucursal. 
    		\item {\texttt{modificaEmpleado()}}.En este método se puede modificar alguno de los atributos de la entidad empleado utilizando el número de identificador del empleado.
    		\item {\texttt{buscaPorEmpleado()}}.Este método regresará el número de sucursal utilizando el numero del empleado.
    	\end{itemize}
    	
    	\item {\bf{Menu}}\\
    	
    	Esta clase es necesaria para la interacción entre el usuario y la aplicación desde la terminal. 
    	
    	\begin{itemize}
    		
    		\item {\texttt{guardaBD()}}. Guarda la nueva información proporcionada en los archivos de la base de datos.
    		\item {\texttt{leerBD()}}. Carga la información de la base de datos en dos tablas hash  
    		para después asignarselas a un objeto de la clase Manejador.
    		\item {\texttt{iniciaMenu()}}.Este método lee los archivos de la base de datos, después inicia el menú principal y guarda la información.
    		\item {\texttt{menuPrincipal()}}. El método se encarga de darle la bienvenida al usuario pidiendole información acerca de que entidad  quiere consultar (Sucursal o Empleado). 
    		\item {\texttt{menuAcciones()}}.Este método recibe el parámetro que se intrudujo en el menú principal y despliega un menú de opciones acerca de las acciones que desea realizar, ya sea el de  agregar, eliminar, modificar o buscar.
    		\item {\texttt{menuElimina()}}. Recibe el parámetro proporcionado en el menú principal para poder eliminar la información de una sucursal o de un empleado, para eso es necesario pedirle al usuario el numero de sucursal o el numero de empleado respectivamente.
    		\item {\texttt{menuAgrega()}}. Se le pide al usuario que ingrese la información con un formato especifico para poder agregar una nueva instancia de la entidad deseada.
    		\item {\texttt{menuModifica()}}. El usuario debe elegir que atributo desea modificar ingresando el número de identificador de acuerdo a la entidad elegida e ingresar la nueva información.
    		\item {\texttt{menuBuscar()}}. Simplemente se muestra un mensaje que le pide al usuario que ingrese el numero de empleado para después regresar el número de sucursal a la que pertenece.
    	\end{itemize}
    	
    	\item {\bf{IO}}\\
    	
    	Esta clase se encarga de leer y escribir los archivos u objetos en formato CSV. Por lo tanto existen dos métodos. Esta el método {\texttt{leer}} que lee un archivo CSV y devuelve una tabla hash que contiene la información y también se encuentra el método {\texttt{escribir}} que escribe un archivo CSV con los datos proporcionados.\\
    	
    \end{enumerate} 

    
    
    \noindent Se menciona 5 diferencias entre almacenar la información mediante un sitema de archivos a almacenarla en una base de datos:
    
    \begin{itemize}
    	\item Mientras mas entidades se tengan mas complicado se vuelve el manejo de las referencias.
    	\item Inenficiencia para realizar consultas.
    \end{itemize}
    
\end{document}
